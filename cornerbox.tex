
\begin{cornerbox}
In the summer of 2018, I had the privilege of contributing a white paper to the Committee on Reproducibility and Replicability in Science of the National Academies of Science on the history and state of reproducibility in economics. In many sciences, new preprint services have emerged within the last two years, e.g., \href{https://psyarxiv.com/}{PsyArXiv}. These are considered to be part of the broader move to greater research transparency. While writing the white paper, I pointed out that this kind of pre-publication exchange has long been the norm in economics. The first National Bureau of Economic Research (NBER) working paper, one of the most prestigious working paper series in economics, was published (in paper form) in 1973 \citep{WelchEducationInformationEfficiency1973}. By the early 1990s, there was a wide variety of such working paper series, typically provided by academic departments and research institutions. Since grey literature at the time was not cataloged or indexed by most bibliographic indexes, a distinct effort to identify both working papers and the novel electronic versions grew from modest beginnings in 1992 at Université de Montréal and elsewhere into what is today known as the Research Papers in Economics (RePEc) network, a “collaborative effort by hundreds of volunteers in 99 countries” \citep{RePEcResearchPapers,KrichelEconomicsOpenBibliographic2009,Batiz-Lazobriefbusinesshistory2012}. The initial index was split into electronic (WoPEc) \citep{KrichelWoPEcElectronicWorking1997}  and printed working papers (BibEc) \citep{KrichelEconomicsOpenBibliographic2009,CruzCatalogingEconomicsPreprints2000}, testimony to the prevalence of the exchange of scientific research in semi-organized ways.  In 1997, BibEc counted 34,000 working papers from 368 working paper series (30). RePEc today has data from around 4,600 working paper series and claims about 2.5 million full-text (free) research items, provided in a decentralized fashion by about 2,000 archives (31). These items not only include traditional research papers, but also, since 1994, computer code (32–34). 
\end{cornerbox}
